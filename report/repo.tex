\documentclass[11pt,a4j,titlepage]{jsarticle}
\usepackage[dvipdfmx]{graphicx}
\usepackage{amsmath, amssymb}
\usepackage{float}
\usepackage{multirow}
\usepackage{url}
\usepackage{subcaption}
\usepackage{tabularx}
\usepackage{listings, jlisting}
\usepackage{here}
\lstset{
basicstyle={\small\ttfamily},
identifierstyle={\small},
keywordstyle={\small\bfseries},
ndkeywordstyle={\small},
stringstyle={\small\ttfamily},
frame={tb},
breaklines=true,
columns=[l]{fullflexible},
numbers=left,
xrightmargin=0zw,
xleftmargin=3zw,
numberstyle={\scriptsize},
stepnumber=1,
numbersep=1zw,
lineskip=-0.5ex
}
\begin{document}
\title{ネットワーク実験レポート課題}
\author{学正番号:09B21601 氏名:西澤陽}
\date{提出日//}
\maketitle
\section{プログラムの処理方針と作成方針}
本実験で作成プログラムはサーバー側とクライアント側でデータの通信を行うプログラムである.
\subsection{作成方針}
本プログラムではサーバー側とクライアント側を分けて別々に作成する.初めにクライアント側について作成する.目標はWebサーバーに対して,HHTPを用いて接続し,Webコンテンツを取得することである.

クライアント側が完成したら,サーバー側を作成する.目標は既に作成したクライアントと1度に一つの通信可能なサーバーを作成することである.

サーバーとクライアントが完成したら,名簿管理プログラムをサーバー,クライアント通信へ拡張する.通信の概要はサーバー側で名簿プログラムを動かし,クライアントから送られてくるコマンドをサーバー側で実行し,結果をクライアントに返すシステムである.
\subsection{処理方針}
本プログラムの処理方針について述べる.初めにクライアントで通信相手の情報を取得し,情報からソケットを作成する.
その間にサーバー側では待ち受けソケットを作成し.ソケットを待ち受け設定にする.サーバー側で待ち受けを開始する.その後,クライアントがTCPコネクション確立を試みる.それに対して,サーバーはコネクションを受け入れ,クライアントとサーバー同士でデータの送受信を行う.最後に共にデータ通信ソケットを終了させ,通信が終了する.

\section{プログラム処理の説明}
\section{プログラムの使用方法と使用例}
\section{プログラムの作成過程に関する考察}
\section{得られた}
\end{document}
